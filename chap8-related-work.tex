\section{Related Work}
\label{sec:related-work}

% Tamper-evident logging
% ----------------------
Tamper-evident logging\cite{ht} allows auditors to ensure a log's correct behavior.
A \emph{history tree} is used to store events in the log, check their membership and prove that a new version of the log is consistent with a past version (i.e., no past events have been removed or modified).
Unfortunately, tamper-evident logging does not address equivocation attacks, assuming auditors can gossip to detect forks.
\Sys offers the same semantics as tamper-evident logging (\ie membership proofs, consistency proofs) but also prevents equivocation.
However, because the Bitcoin blockchain is implemented as a hash-chain, \Sys's membership and consistency proofs are linear, not logarithmic, in the log's size.
In practice, a \Sys log can commit root hashes of a history tree and prevent equivocation about the tree, while preserving logarithmic membership with respect to a root hash and logarithmic consistency proofs between two consecutive root hashes.

% Proofs of proofs of work
% ------------------------
Proofs of proofs of work (PPOW) \cite{ppow} enable thin clients to verify the ``weight'' of a blockchain (see \secref{sec:background:bitcoin:consensus}) by downloading only $O(\log{n})$ rather than all $n$ block headers.
Should Bitcoin adopt this interesting technique, \Sys clients could leverage it to download fewer block headers from the HRN.
First, we can leverage PPOWs to decrease the bootstrapping bandwidth of \Sys clients from $O(n)$ to $O(\log{n})$ when initially syncing with a blockchain of $n$ blocks.
Second, we can leverage PPOWs to skip downloading some of the block headers between two statements $s_i$ and $s_{i+1}$, should there be a large number of blocks between consecutive statements.
However, we cannot leverage PPOWs to skip downloading \Sys transactions and their corresponding block headers.
This is because \Sys clients need to verify that all \Sys transactions are chained correctly (see \secref{sec:catena:design:auditing}).

% uniquebits and CommitCoin
% -------------------------
Peter Todd's uniquebits\cite{uniquebits} allows users to publish signed hashes of arbitrary data in the Bitcoin blockchain for later auditing.
At a high level, the scheme commits the signed data $d$ and PGP fingerprint information about the signer by leveraging both ``Pay-to-Script Hash'' (P2SH) transactions\cite{bitcoin-p2sh-data} and ``fake'' \pks (see \secref{sec:background:bitcoin:opret}).
Unfortunately, uniquebits does not address the equivocation problem: the scheme cannot efficiently prevent a signer from publishing two pieces of data $d$ and $d'$ and equivocating to thin clients about which piece was signed.
Similarly, CommitCoin\cite{commitcoin} uses the Bitcoin blockchain to ``timestamp'' commitments and prove they were made at a certain time, but cannot efficiently prevent equivocation.

% Single-use seals
% ----------------
% Genesis TXN:
% https://blockchain.info/tx/b8b86d41647db9eb4fcb833cc685eb1264d7c3d2b8fb056551b3388e8c35a43b?show_adv=true
% Spends genesis, commits ID of next TXN (i.e., 5b6705437ced5cca8ec52cfb187e5d3270e067817c4fcd5a26755d4ceb4c010c)
% https://blockchain.info/tx/976e842f944b797352a111893d766ce1b3b211d987d31e764a424ee19de10fe5?show_adv=true
% Spends some random, unrelated transaction
% https://blockchain.info/tx/5b6705437ced5cca8ec52cfb187e5d3270e067817c4fcd5a26755d4ceb4c010c?show_adv=true
% Spends transaction above, , commits ID of next TXN
% https://blockchain.info/tx/19c2aa81cf6a0bf7d1f20810ca72baac44cb2532d765e162b0d58b4e050ff3af?show_adv=true
Concomitant with the publication of our online version of \Sys\cite{catena-eprint}, Peter Todd independently proposed a Bitcoin-based log that prevents equivocation using \emph{single-use seals}\cite{single-use-seals}.
Single-use seals generalize the concept of spending a transaction output in Bitcoin while committing some arbitrary data in the process.
The transaction output waiting to be spent is the \emph{identifier} of the single-use seal, while the transaction that spends that output and commits some arbitrary data (see \secref{sec:background:bitcoin:opret}) is known as the \emph{closing transaction}.
Importantly, when a seal is ``closed'' by spending its output via a closing transaction, another seal can be specified as the arbitrary data in that transaction.
This is similar to how, in \Sys, a new transaction spends the previous transaction's continuation output (equivalent to ``closing a seal'') and creates a new continuation output (equivalent to specifying a new seal).

As opposed to \Sys, single-use seals provide a certain degree of censorship-resistance because the identifier of the next seal can be hidden from miners by committing to it in the closing transaction.
In contrast, with \Sys, the next continuation output (see \secref{sec:catena:design:transactions}) is always public and known by miners, so they can censor transactions that try to spend it.
However, \Sys is more efficient than single-use seals, since each statement requires a single transaction to be posted on the blockchain, while single-use seals require two transactions (one transaction for the seal's identifier plus another closing transaction).
Furthermore, although thin client verification of a log would be possible with single-use seals, the details of this are never considered in depth\cite{single-use-seals}, a contribution that \Sys makes.
Finally, to the best of our knowledge, single-use seals have not been implemented yet, making \Sys the first implementation of an efficient Bitcoin-based log.

% Keybase
% -------
Keybase\cite{keybase} and Blockstack\cite{blockstack,virtualchain} use the Bitcoin blockchain to prevent equivocation but do so inefficiently, requiring clients to run a full Bitcoin node.
Keybase periodically publishes the Merkle root of its \pkd by committing it in transactions signed under a fixed \pk known by Keybase clients\cite{keybase-txs}.
Specifically, Keybase stores the Merkle root in the transaction's output as a ``fake'' \pk (see \secref{sec:background:bitcoin:opret}).
Unfortunately, this approach does not allow clients to efficiently and correctly obtain all Keybase-issued transactions using Bloom filtering (see \secref{sec:background:bitcoin:thin}).
This is because Bitcoin P2P nodes can selectively hide transactions from clients and show different transactions to different users, thus equivocating about the Keybase directory (see \secref{sec:catena:design:auditing}).
In \Sys, a new transaction always spends the previous one, creating a unique chain due to the difficulty of double spending.
As a result, all \Sys clients see the same history of transactions, implicitly agreeing on the history of statements.

% Blockstack
% ----------
In Blockstack\cite{blockstack}, users submit their own operations (e.g., ``register \pk'' or ``update \pk'') to the Bitcoin blockchain by creating transactions that include these operations in an \opret output.
Blockstack nodes download the full Bitcoin blockchain and filter these Blockstack-specific transactions (recognized via a magic byte in the \opret data), accumulating the Blockstack operations into a \emph{consensus hash}\cite{blockstack}.
Importantly, thin clients can query Blockstack nodes for \pks (\eg ``look up Alice's \pk'') and authenticate their responses against a consensus hash.
Unfortunately, because thin clients cannot download the entire blockchain, they would have to obtain consensus hashes from a trusted entity which, if compromised, could equivocate about these hashes.
In this sense, \Sys could make equivocation about consensus hashes as hard as forking Bitcoin, thereby increasing Blockstack's thin client security (discussed in \secref{sec:background:motivation:blockchain-transparency}).

% Colored coins
% -------------
Coin coloring schemes \cite{colu,openassets,coinspark} leverage the Bitcoin blockchain to enable the secure issuance and transfer of assets different than bitcoins, such as smart property, stocks or bonds\cite{coloredcoinsoverview}.
The key idea is that a set of coins locked in an output can be assigned a static color which can be correctly maintained as those coins change hands.
To prevent double spending of colored coins, coloring schemes also leverage Bitcoin's security against double spends along with some additional verification by ``color-aware'' wallets.
The overhead of these schemes is similar to \Sys's, since ``color-aware'' (thin) wallets only need to keep track of transactions that affect an asset (e.g., reassigned ownership).

Coin coloring schemes and \Sys both rely on the difficulty of double spending but do so to solve slightly different problems.
Specifically, while coin coloring schemes prevent equivocation about the never-changing color of a coin, \Sys prevents equivocation about an ever-growing log of statements.
While some coloring schemes support committing arbitrary data in their transactions\cite{coloredcoins-metadata} and could be adapted to prevent equivocation, to the best of our knowledge, this has not yet been done.
Thus, we believe \Sys to be the first system that solves the equivocation problem efficiently using Bitcoin.

% Coins on fire!
% --------------------------
Non-equivocation contracts\cite{liarliar} disincentivize equivocation by penalizing it with monetary loss.
Specifically, if an authority equivocates, its Bitcoin secret key, which locks some funds, is implicitly revealed via a mechanism similar to double-authentication-preventing signatures (DAPS)\cite{daps}.
As a result, anybody who detects equivocation can spend those funds.
The advantage of non-equivocation contracts is that statements are not included in the Bitcoin blockchain so they can be issued faster than in \Sys.
This enables interesting applications, such as asynchronous payment channels\cite{liarliar}, which are not possible with \Sys.
However, as opposed to \Sys, non-equivocation contracts only disincentivize equivocation and do not necessarily prevent it.
For example, an outsider who steals the authority's secret key and wants to harm the authority is actually incentivized to equivocate and can easily do so.
With \Sys, even with a stolen secret key, an outsider cannot equivocate without forking the Bitcoin blockchain.

% CoSi
% ----
CoSi\cite{cosi} prevents equivocation by requiring a threshold number of ``witnesses'' to also verify and sign an authority's statements.
Depending on the application, these witnesses could also check the internal consistency of statements (see \secref{sec:discussion:agnostic}). 
In contrast, \Sys prevents equivocation by assuming its authority, Bitcoin, is trustworthy, without relying too much on the header relay network (HRN) to keep it honest (see \secref{sec:attacks:hrn}).
Both CoSi and \Sys make assumptions about connectivity of participants.
CoSi requires a relatively well-connected set of witnesses for its tree broadcast scheme while \Sys requires the Bitcoin P2P network not to be easily partitioned.
One drawback of CoSi is that it requires an admission control process for witnesses to prevent Sybil attacks\cite{sybil}.
As a result, finding witnesses who are reputable, trustworthy entities could be hard.
In contrast, \Sys could be easier to deploy since it only relies on the Bitcoin blockchain as a single trustworthy witness and on a header relay network that can be bootstrapped using existing blockchain explorers (see \secref{sec:catena:design:header-relay}).

Like CoSi, \Sys can offer both ``proactive'' and ``retroactive'' security \cite{cosi}.
In particular, \Sys can be used retroactively by clients to validate previously accepted statements, or it can be used proactively before accepting a statement as valid.
Unlike CoSi, \Sys suffers a higher delay when used proactively because clients have to wait for sufficient confirmations before accepting a statement.
This is a cost \Sys pays for using a decentralized consensus network as its only witness.
However, we stress that not all applications care about this cost.
In particular, \Sys is suitable for key transparency schemes, Tor directory servers and software transparency schemes, which all perform batching and update their state infrequently.

% Key transparency
% ----------------
Key transparency schemes can detect equivocation using gossip amongst users\cite{ctgossip,ect,coniks,dtki}, gossip between users and trusted validators\cite{aki}, federated trust\cite{coniks}, any-trust assumptions\cite{arpki} or non-collusion between actors\cite{arpki, aki}.
\Sys instead relies on the resilience of Bitcoin's proof-of-work consensus to prevent, not just detect, equivocation.
Our approach can provide proactive security\cite{cosi} at the cost of publishing new statements every 10 minutes with an average 60-minute confirmation latency (if clients wait for 6 confirmations).
Alternatively, \Sys can provide retroactive security with no latency.
We believe \Sys can strengthen key transparency schemes because it enables anyone to audit efficiently for non-equivocation.
We also believe \Sys's approach to non-equivocation is simpler and more trustworthy due to the decentralized nature of Bitcoin's consensus protocol.

% EthiKS
% ------
% 1 gas is usually 20 gweis
% 1 gwei is 0.000000001 = 10^-9 ETH
% 1 gas = 20 gwei = 0.0000 0002 ETH = 2 * 10^-8 ETH
% 1 ETH = 12 USD
% 1 gas = 2 * 10^-8 * 12 USD
%
% EthIKS costs|
% ------------*------------*----------------------------
% Create tree | 367535 gas | 0.01837675 USD = 1.8 cents
% Insert user | 42042 gas  | 0.0021021 USD = .2 cents
% Update user | 12042 gas  | 0.0006021 USD = .06 cents
EthIKS\cite{ethiks} uses the Ethereum blockchain\cite{ethereum} to prevent equivocation in CONIKS\cite{coniks}, a key transparency scheme that enables users to efficiently monitor their own public key bindings.
EthIKS implements CONIKS as a ``smart contract'' in the Ethereum blockchain and relies on Ethereum miners to enforce CONIKS security invariants.
Like \Sys, EthIKS also efficiently leverages proof-of-work consensus within a cryptocurrency to prevent equivocation in the log.
However, different from \Sys, EthIKS guarantees internal consistency (see \secref{sec:discussion:agnostic}) of the CONIKS log, though this comes at the expense of additional Ethereum transaction fees paid by the EthIKS log server\cite{ethiks}.
In contrast, \Sys clients have to check each log statement for internal consistency themselves, incurring an overhead linear in the number of statements in the log.
For certain applications where internal consistency checks are not expensive (e.g., monitoring your own binding in a \pkd) and minimizing server costs is a priority, \Sys could be better suited.
However, when server costs are not a concern (e.g., costs can be shifted to users), Ethereum-based approaches like EthIKS could be better suited.

\section{Background and Motivation}
\label{sec:background}
In this section, we discuss our motivation for designing \Sys and give the necessary background on Bitcoin needed to understand \Sys's design.

\subsection{Motivation}
\label{sec:background:motivation}
Our main motivation for designing \Sys is to provide proactive security to many applications that depend on it.
At the same time, we want to improve previous blockchain-based transparency schemes\cite{keybase,virtualchain} whose shortcomings we describe in \secref{sec:background:motivation:blockchain-transparency}.
Finally, we want a non-equivocation scheme that does not require many trustworthy parties to come into existence and that can be deployed today.

\subsubsection{Key Transparency}
\label{sec:background:motivation:key-transparency}
% NOTE: Under their threat models (e.g., gossip works), they also prevent equivocation.
% .ECT: Section 2.2, page 3
% .DTKI: Section 6, page 12
% .CONIKS: Section 2.1, "Auditors", page 4
% .AKI: 
%   Section 4, page 3, Figure 1: browsers check roots against validators.
%   Section 2.2, page 2: non-collusion between CAs, log servers and validators. EFF is a validator.
% .ARPKI:
%   Section 4, page 4: Non-equivocation? (security) still holds even when n - 1 entities are compromised. Validators are optional (CAs are validators).
\Sys can prevent equivocation attacks in current key transparency work\cite{ct,ect,dtki,arpki,aki,coniks} and, as a result, thwart man-in-the-middle (MITM) attacks.
Key transparency schemes bundle public key bindings together into a directory implemented using authenticated data structures\cite{ad}.
Users are presented with digests of the directory as it evolves over time and can verify someone's \pk against a digest of the directory, preventing equivocation with respect to that digest.
The remaining problem for key transparency schemes is to prevent equivocation about the digests themselves.
For this, current schemes rely on federated trust\cite{coniks}, any-trust assumptions\cite{arpki}, non-collusion between actors\cite{arpki, aki} or on users gossiping between themselves\cite{ctgossip,ect,coniks,dtki} or with trusted validators\cite{aki}.

With \Sys, we propose using the Bitcoin blockchain as a hard-to-coerce, trustworthy witness that can vouch for directory digests.
For example, in Certificate Transparency (CT), a log server would directly witness \emph{signed tree heads} (STHs) in Bitcoin via a \Sys log.
Users can efficiently look up new STHs in the \Sys log and be certain that the log server has not equivocated about them.
We believe this approach could be more resilient to attacks, as a compromised log server cannot equivocate without forking the Bitcoin blockchain.
Also, because most transparency schemes publish digests of the directory periodically, we believe they are amenable to being secured by \Sys.

\subsubsection{Blockchain-based Transparency}
\label{sec:background:motivation:blockchain-transparency}
Blockchain-based transparency schemes\cite{keybase,blockstack} are promising due to their simplicity and resilience to forks, but the overhead of downloading all blockchain data makes them unusable on many devices.
\Sys can decrease the overhead of these schemes from currently \blockchainsize\cite{bitcoin-size} to around \headerssize.
For example, \Sys can enable thin clients running on mobile phones to efficiently audit the Bitcoin-witnessed Keybase \pkd\cite{keybase}.
Currently, Keybase publishes digests of their \pkd in Bitcoin by creating transactions signed by a predetermined \pk\cite{keybase-scheme}.
Keybase clients recognize these transactions and read directory digests from them (see \secref{sec:related-work} for details).
The problem with this approach is that thin clients cannot securely use Bloom filtering (see \secref{sec:background:bitcoin:thin}) to avoid downloading irrelevant transactions, as an adversary could selectively hide Keybase transactions and equivocate about the directory (we explain this attack in \secref{sec:catena:design:auditing}).
\Sys prevents this attack and also has the advantage of not polluting Bitcoin's unspent transaction output (UTXO) set\cite{keybase-opret}.

\Sys can also be used to improve Blockstack's thin client security \cite{blockstack}.
Currently, to benefit from Bitcoin's resilience against forks, Blockstack clients need to download the entire blockchain and compute their own \emph{consensus hash} over all Blockstack-related operations (see \secref{sec:related-work} for details).
Blockstack clients could also choose to trust someone else's consensus hash and verify \pk lookups against it efficiently using Simplified Name Verification (SNV)\cite{blockstack}.
However, clients still have to download full Bitcoin blocks to update that consensus hash or continue trusting someone else to update it.
As with Keybase, Bloom filtering cannot be used securely to filter Blockstack transactions.
To fix this problem, we propose using a \Sys log to keep track of Blockstack operations rather than scattering them through the blockchain.
In this way, thin clients can efficiently download just the Blockstack operations and quickly compute their own consensus hashes.

One disadvantage of this approach, according to one of the Blockstack co-founders\cite{blockstack-shea-jun-2016}, is that it requires a secret key to manage the \Sys log and would thus ``centralize'' the system.
To address this, an alternative design would be to introduce \emph{auditors} who verify and publish Blockstack consensus hashes in a jointly-signed \Sys log.
While this approach centralizes trust for thin clients, such as mobile phones, it does so in a more accountable and transparent manner.
Specifically, the auditors can't equivocate about consensus hashes but can still publish \emph{internally inconsistent}\cite{ht} consensus hashes (see \secref{sec:discussion:agnostic}).
However, such misbehavior would be evident in the Bitcoin blockchain when audited by a full Blockstack client.

\subsubsection{Software Transparency}
\Sys can prevent equivocation in \emph{software transparency} schemes\cite{software-transparency} and thus thwart man-in-the-middle attacks that try to inject malicious software binaries on victims' machines\cite{software-transparency}.
In fact, Bitcoin developers were concerned in the past about these kinds of attacks on Bitcoin binaries\cite{bitcoin-binary-transparency}.
To prevent these attacks, software vendors can publish digests of new versions of their software in a \Sys log.
Customers can then verify any version downloaded from a vendor's website against the vendor's log.
Previous work\cite{cosi} already highlights the necessity of software transparency in the face of insecure software update schemes\cite{secure-software-updates,attacks-on-package-managers}, key loss or compromise \cite{microsoft-golden-key} and black markets for code-signing certificates\cite{blackmarket-certs}.

\subsubsection{Tor Directory Servers}
\Sys can be used to prevent Tor directory servers\cite{tor} from equivocating about the directory of Tor relays.
Equivocation attacks are particularly concerning for Tor because they enable an attacker to easily deanonymize users by pointing them towards attacker-controlled Tor relays.
In fact, Tor Transparency\cite{tortransparency} plans to address these attacks by publicly logging the Tor directory consensus.
In the same spirit, we propose using \Sys to increase the resilience of Tor Transparency.
With \Sys, directory servers can publish the consensus in a \Sys log by jointly signing it using a Bitcoin multisignature\cite{multisig}.
Since Tor does not try to conceal who is connected to the network\cite{tor}, we are not concerned about \Sys's header relay network learning who is using Tor.
% If Tor users want to hide that they use Tor, they can rely on Tor bridges.
Finally, because Tor consensus is updated every hour, we believe it should be suitable for embedding in a \Sys log.

\subsubsection{Consensus Amongst $n$ Servers}
\Sys can be used by a set of $n$ servers to reach consensus on a log of operations, where each server manages its own secret key and does not necessarily trust the other $n-1$ servers.
In this scheme, each server submits an operation to the log by creating a \Sys transaction that is spendable by all $n$ servers (see \secref{sec:model:actors:log-server}).
To disincentivize the other servers from stealing the coins, the log is funded with small amounts of bitcoins and is frequently ``re-funded'' (see \secref{sec:catena:design:refund}).
This scheme allows all servers to reach consensus on the log and relies on Bitcoin miners to decide which server's operation gets included in the log.
To prevent adversarial servers from monopolizing the log with their operations by paying higher transaction fees, the servers can agree on an upper bound on fees.

%\subsubsection{Integrity-sensitive Systems}
%Verena, SUNDR\cite{sundrosdi}, SPORC. SPORC/SUNDR assume clients know each other's public keys (e.g., PKI) => clients can gossip securely and detect equivocation.

\subsection{Bitcoin Background}
\label{sec:background:bitcoin}

Bitcoin\cite{bitcoin,sokbitcoin,princetonbitcoinbook,bitcoinandbeyond} is a peer-to-peer digital currency that allows users to mint digital coins called \emph{bitcoins} and exchange them without a trusted intermediary.
Bitcoin uses a novel permissionless Byzantine consensus protocol known as \emph{proof-of-work consensus}\cite{blockchainproto} which allows all participants to agree on a log of transactions and prevent attacks such as double spending coins.
The log of transactions is called a \emph{blockchain} and is stored and managed by a peer-to-peer (P2P) network\cite{bitcoin-p2p}.
A special set of users called \emph{miners} run Bitcoin's proof-of-work consensus protocol, extending the blockchain with new \emph{blocks} made up of new transactions.
This process, called \emph{mining}, is computationally difficult and secures Bitcoin by allowing everyone to agree on the correct log of transactions while preventing Sybil attacks\cite{sybil}.
To incentivize Bitcoin miners to mine, a \emph{block reward} consisting of newly minted bitcoins is given to a miner if he mines or ``finds'' the next block.

\subsubsection{P2P Network}
\label{sec:background:bitcoin:p2p}
Bitcoin uses a peer-to-peer (P2P) network of volunteer nodes to store the blockchain\cite{bitcoin-p2p}, listen for new transactions or new blocks, and propagate this information throughout the network.
Users, such as merchants and their customers, download the blockchain by becoming part of the P2P network and can then receive or issue Bitcoin transactions.
Miners are also part of the P2P network where they listen for new blocks and broadcast their own blocks.

\subsubsection{Blockchain}
\label{sec:background:bitcoin:blockchain}
Bitcoin's ``blockchain'' is implemented as a hash-chain of \emph{blocks} (see \figref{fig:bitcoin}) and keeps track of all transactions in the system, allowing anyone to verify that no double spends have occurred.
A Bitcoin block is made up of a set of transactions (up to 1 MB) and a small \emph{block header} (80 bytes) that contains a hash pointer to the previous block.
The transactions in the block are hashed in a Merkle tree\cite{merkle} whose root hash is stored inside the block header.
The Merkle tree allows Bitcoin \emph{thin clients} (see \secref{sec:background:bitcoin:thin}) to obtain efficient \emph{membership proofs} that a transaction is part of a block.

% or t for top, b for bottom, h for here (https://tex.stackexchange.com/questions/35125/how-to-use-the-placement-options-t-h-with-figures)
\begin{figure}[t]
    \centering
    \includegraphics[trim=.8cm .8cm 1.4cm .8cm, width=1\columnwidth]{figs/bitcoin.pdf}
    \vspace{-1.5cm}
    \caption{The Bitcoin blockchain is a hash chain of blocks. Each block has a Merkle tree of transactions. Efficient membership proofs of transactions can be constructed with respect to the Merkle root. Here, $tx_1$ transfers coins from Alice, Bob and Carol to Dan and somebody else (miners receive a fee of 1 coin).
        Alice authorizes the transfer of her coins by signing $tx_1$, which has an input pointing to her coins locked in the 1st output of $tx_a$.
        Bob and Carol do the same.
        Similarly, Dan later spends his coins locked in $tx_1$'s 1st output by signing a new transaction $tx_d$ with an input pointing to $tx_1$'s 1st output.}
    \label{fig:bitcoin}
\end{figure}

\subsubsection{Decentralized Consensus}
\label{sec:background:bitcoin:consensus}
To solve the consensus problem in the decentralized or \emph{permissionless} setting, where participants can enter and leave the protocol as they please, Bitcoin introduces a novel Byzantine consensus protocol called \emph{proof-of-work consensus}\cite{blockchainproto,bitcoin-backbone,bitcoin-speed-security,miller2014anonymous}.
Though it does so at a high computational cost, this protocol defeats Sybil attacks\cite{sybil} and achieves consensus on the blockchain if 51\% of the computational power amongst participants remains honest.

Participants called \emph{miners} race to solve computationally-difficult proof-of-work puzzles derived from the previous Bitcoin block.
If a miner finds a solution, the miner can publish the next block by announcing it along with the solution (in reality, the solution is part of the next block) over the P2P network.
Furthermore, this miner will receive a \emph{block reward} in bitcoins, an incentive for miners to participate in the consensus protocol.
The puzzle difficulty is adjusted every 2016 blocks based on the inferred computational power of the miners, or \emph{network hashrate}, so that a new block is found or ``mined'' on average every 10 minutes.

When two miners find a solution at the same time, the Bitcoin blockchain is said to \emph{accidentally fork} into two chains.
In this case, Bitcoin peers use the \emph{heaviest chain rule} and select the heavier fork as the \emph{main chain} that dictates consensus.
The \emph{weight} of a fork is simply the amount of computational work expended to create that fork.
Assuming no difficulty changes, the heaviest fork is the longest fork.
However, across difficulty changes, it could be that a fork with fewer blocks is heavier than a longer fork (though this never happens in practice).

During an accidental fork, both forks have the same length and weight (assuming the fork does not cross a difficulty recomputation point), so Bitcoin peers adopt the fork they saw first as their main chain.
As more blocks are mined, one of the forks becomes heavier than the other and is accepted as the main chain by the whole network\cite{blockchainproto}.
In this case, the other abandoned fork and its blocks are said to be ``orphaned.''
In practice, accidental forks are infrequent and short: no more than one or two blocks get orphaned.
To deal with accidental but also with malicious forks, most Bitcoin nodes only consider a block and its transactions \emph{confirmed} if 6 or more blocks have been mined after it.

\subsubsection{Transactions}
\label{sec:background:bitcoin:transactions}
Bitcoin transactions facilitate the transfer of coins between users (see \figref{fig:bitcoin}).
A Bitcoin transaction has an arbitrary number of \emph{transaction inputs}, which authorize the transfer of coins, and \emph{transaction outputs} (TXOs), which specify  who receives those coins and in what amounts.
Naturally, the number of coins locked in the outputs cannot exceed the number of coins specified in the inputs (with the exception of so-called ``coinbase'' transactions, which mint new coins and have no inputs).
A transaction output specifies an amount of coins and their new owner, most commonly as a \pk.
A transaction input refers to or ``spends'' a previously unspent transaction output (UTXO) and contains a proof-of-ownership from that UTXO's owner, which authorizes the transfer of those coins.
For the purposes of this paper, we only make use of the case where outputs specify owners using \pks and inputs prove ownership using signatures.

Importantly, when assembling transactions into blocks, Bitcoin miners prevent double spends by ensuring that, across all transactions in the blockchain, for every TXO there exists at most one transaction input that refers to or spends that TXO.
This invariant is known as the \emph{TXO invariant} and \Sys leverages it to prevent forks.
Finally, a transaction's fee is the difference between the coins spent in its inputs and the coins transferred by its outputs.
The fee is awarded to the miner who mines a block containing that transaction.
In theory, the fee can be zero, but in practice recent contention for space in the blockchain requires users to pay transaction fees.

\subsubsection{Storing Data in Transactions}
\label{sec:background:bitcoin:opret}
Bitcoin allows users to store up to 80 bytes of data in transactions through provably-unspendable \opret transaction outputs.
Importantly, any coins specified in the output are forever unspendable or ``burned''.
For simplicity, \Sys uses \opret outputs to store application-specific statements in the Bitcoin blockchain (see \secref{sec:catena:design}).
However, there are other ways to store data in Bitcoin transactions: in the value of transferred coins\cite{bitcoin-storing-data}, in transaction inputs \cite{bitcoin-p2sh-data}, in transaction sequence numbers\cite{bitcoin-storing-data}, or in an output's \pk (either via vanity \pks\cite{bitcoin-storing-data}, fake \pks\cite{keybase-scheme}, multisig \pks\cite{multisig} or ``pay-to-contract'' \pks\cite{bitcoin-pay-to-contract}).

\subsubsection{Thin Nodes vs. Full Nodes}
\label{sec:background:bitcoin:thin}
Bitcoin's P2P network is made up of \emph{full nodes}, which download the entire blockchain and validate all the transactions (see \secref{sec:background:bitcoin:p2p}) and \emph{thin nodes}, which only download small 80 byte block headers and cannot fully validate transactions.
Since full nodes are more expensive to run (higher bandwidth, computation and space), smaller devices such as smartphones can run thin nodes instead, also known as \emph{Simplified Payment Verification} (SPV) nodes.

Thin nodes verify Bitcoin transactions more efficiently under a slightly stronger assumption about the Bitcoin network.
A thin node considers a transaction valid if it sees a correct Merkle proof of membership for that transaction in a block.
Furthermore, the more blocks are mined after a transaction's block (also known as \emph{confirmations}), the more confident a thin node can be that the transaction is indeed valid.
Importantly, thin nodes don't even verify signatures on transactions: the membership proof coupled with enough confirmations offers enough assurance that the transaction was verified by miners and is thus valid.
As a result, thin nodes assume Bitcoin miners follow their incentives and create correct blocks or otherwise thin nodes could accept invalid transactions.
This assumption can be reasonable since miners would lose their block reward if they create invalid blocks (see \secref{sec:background:bitcoin:consensus}).

Finally, the only way for thin nodes to avoid downloading unnecessary data is to use a Bitcoin feature called Bloom filtering\cite{bitcoin-bloom}.
This feature allows thin nodes to only receive transactions of interest by asking remote peers to filter out irrelevant transactions using a Bloom filter\cite{bloom}.
Bloom filtering is cheap for the requesting thin client but quite expensive for the servicing full node, which has to load all requested blocks from disk, pass them through the filter and send filtered blocks to the thin client.
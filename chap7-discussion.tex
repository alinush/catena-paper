\section{Discussion and Future Work}
\label{sec:discussion}

\setcounter{subsubsection}{0}
\subsubsection{Building \Sys on Top of Bitcoin}
We chose to design \Sys on top of Bitcoin because of Bitcoin's resilient proof-of-work consensus\cite{blockchainproto} and its real-world deployment.
This makes \Sys-enabled applications easy to deploy (no need to wait for trustworthy parties to come into existence) and expensive to attack (adversaries have to double spend in Bitcoin).
Still, it's important to note that Bitcoin's security as a ``black box'' consensus protocol remains an open problem.
For example, it can be difficult to dismiss externally-motivated adversaries who are well-incentivized to maliciously mine and double spend.
Finally, we note that \Sys could also be built on top of other blockchains such as Ethereum\cite{ethereum}, but we believe Bitcoin's security currently outmatches the security of all other blockchains.

% Regarding SPV, Peercoin doesn't seem to support it: https://talk.peercoin.net/t/does-peercoin-support-simplified-payment-verification/3697
% ...and neither Blackcoin nor NXT seem to mention anything about SPV.
In particular, we avoided Ethereum for a few reasons.
First, we believe Bitcoin is a more mature ecosystem to base applications on, given the many blockchain-based apps built on top of it\cite{blockstack,keybase,openassets,colu,coinspark}.
Second, Ethereum plans on transitioning to a proof-of-stake consensus algorithm\cite{proof-of-stake} called Casper\cite{ethereum-casper} that could change the trust assumptions behind thin nodes.
For instance, an additional assumption in Casper is that clients who are offline for too long can authenticate a list of ``bonded'' validators out-of-band\cite{ethereum-casper}.
Third, recent work shows that rational Ethereum miners have an incentive to skip verifying ``expensive'' blocks that other miners constructed maliciously\cite{consensuscomputer}.
In Bitcoin, such attacks are less practical since block verification does not involve executing arbitrarily complex smart contracts.

\subsubsection{Censorship}
\Sys's liveness depends on the censorship-resistance of the Bitcoin network.
Malicious miners can censor \Sys transactions and exclude them from the Bitcoin blockchain, which reduces the liveness of a \Sys log.
We stress, however, that Bitcoin's decentralized consensus does provide some degree of censorship-resistance by allowing \emph{any} honest miner to join the protocol, eventually resulting in an honest, non-censoring, majority.
We also stress that censorship attacks have not been observed in practice and we leave a more careful analysis of Bitcoin's censorship-resistance to future work.

\subsubsection{Historical Consistency}
\label{sec:discussion:agnostic}
\Sys is application-agnostic and does not guarantee application-specific \emph{internal consistency}\cite{ht} of statements, which needs to be checked at the application layer.
Instead, \Sys only guarantees \emph{historical consistency}\cite{ht} of statements, enabling applications to later check the correct semantics of statements.
As an example, \Sys ensures that all clients of a key transparency scheme such as Certificate Transparency (CT)\cite{ct} see the same history of signed tree heads (STHs).
However, clients still have to check the internal consistency of the STHs to detect impersonation.
For instance, Bob's client will want to make sure that across all STHs, his \pk has not been changed maliciously, and thus he hasn't been impersonated.

It is important to understand that without historical consistency, any guarantees of internal consistency are meaningless.
This is exactly why we designed \Sys. 
For instance, a malicious CT log server\cite{ct} can equivocate, giving Alice a signed tree head (STH) with her real \pk and a fake \pk for Bob, while giving Bob a different STH with his real \pk and a fake \pk for Alice.
Alice and Bob both verify their own STHs as being internally consistent and believe they were not impersonated.
However, because Alice and Bob have no historical consistency, they are looking at different STHs, which means the internal consistency guarantees they have are essentially useless.
In this case, Alice and Bob are being impersonated even though internal consistency tells them they are not.
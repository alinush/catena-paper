\section{Conclusion}
\label{sec:conclusion}
We design and implement \Sys, an append-only log that is as hard to fork as the Bitcoin blockchain but efficient to verify by thin clients such as mobile phones.
Specifically, in \Sys, an attacker can equivocate if and only if he can double spend Bitcoin transactions, which is notoriously difficult due to Bitcoin's proof-of-work consensus.
The key idea behind \Sys is to chain \opret transactions together by having a new transaction spend the previous one, making equivocation in the log as hard as double spending in Bitcoin.

\Sys can be used to prevent equivocation in key transparency schemes, paving the way for more trustworthy \pkds.
\Sys can also be used as a public log for Tor Consensus Transparency\cite{tortransparency}, as a software transparency scheme to prevent malicious software updates or as a consensus log for mutually distrusting participants.
\Sys's overheads are small.
Clients only need to download 80-byte block headers and 600-byte statements, a significant improvement over previous blockchain-based transparency schemes\cite{keybase,virtualchain,blockstack} which currently require auditors to download \blockchainsize of blockchain data\cite{bitcoin-size}.
We develop a prototype of \Sys in Java and apply it to CONIKS, a key transparency scheme, demonstrating \Sys's feasibility.
Next, we plan on extending our prototype to scale for popular applications.

Our main reason for designing \Sys is to prevent equivocation in compromised online services.
In that sense, we believe \Sys can bring Bitcoin's non-equivocation guarantees to many important applications today.
In particular, we hope \Sys can be adopted by secure messaging apps such as Signal\cite{signal} or \pkds such as Keybase\cite{keybase}, giving end users stronger guarantees about non-equivocation.
%\subsection*{Abstract}

\begin{abstract}
We present \emph{\Sys}, an efficiently-verifiable Bitcoin witnessing scheme.
\Sys enables any number of thin clients, such as mobile phones, to efficiently agree on a log of application-specific statements managed by an adversarial server.
\Sys implements a log as an \opret transaction chain and prevents forks in the log by leveraging Bitcoin's security against double spends.
Specifically, if a log server wants to equivocate it has to double spend a Bitcoin transaction output.
Thus, \Sys logs are as hard to fork as the Bitcoin blockchain: an adversary without a large fraction of the network's computational power cannot fork Bitcoin and thus cannot fork a \Sys log either.
However, different from previous Bitcoin-based work, \Sys decreases the bandwidth requirements of log auditors from \blockchainsize to only tens of megabytes.
More precisely, our clients only need to download all Bitcoin block headers (currently less than \headerssize) and a small, 600-byte proof for each statement in a block.
We implement \Sys in Java using the \emph{bitcoinj} library and use it to extend CONIKS, a recent key transparency scheme, to witness its \pkd in the Bitcoin blockchain where it can be efficiently verified by auditors.
We show that \Sys can secure many systems today, such as \pkds, Tor directory servers and software transparency schemes.
\end{abstract}
